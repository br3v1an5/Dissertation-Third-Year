\documentclass[final]{beamer}
% beamer 3.10: do NOT use option hyperref={pdfpagelabels=false} !
% \documentclass[final,hyperref={pdfpagelabels=false}]{beamer} 
% beamer 3.07: get rid of beamer warnings

\mode<presentation> {  
%% check http://www-i6.informatik.rwth-aachen.de/~dreuw/latexbeamerposter.php for examples
  \usetheme{Durham} %% This points to the theme cooked up by the final year tutor
}


\usepackage{siunitx}
\usepackage[english]{babel} 
\usepackage[latin1]{inputenc}
\usepackage{amsmath,amsthm, amssymb, latexsym}

  \usefonttheme[onlymath]{serif}
  \boldmath
  \usepackage[orientation=portrait,size=a0,scale=1.4,debug]{beamerposter}                       

  % e.g. for DIN-A0 poster
  % \usepackage[orientation=portrait,size=a1,scale=1.4,grid,debug]{beamerposter}
  % e.g. for DIN-A1 poster, with optional grid and debug output
  % \usepackage[size=custom,width=200,height=120,scale=2,debug]{beamerposter} % e.g. for custom size poster
  % \usepackage[orientation=portrait,size=a0,scale=1.0,printer=rwth-glossy-uv.df]{beamerposter}
  % e.g. for DIN-A0 poster with rwth-glossy-uv printer check ...
  %

  \title[Final Year Project Poster]{Enigma and BOMBE Simulation}
  \author[A L Gillies]{Alexander Gillies}
  \institute[Durham]{School of Engineering and Computing Sciences, Durham University}
  \date{16th April 2012}

  \begin{document}
  \begin{frame}{} 

  \vfill
    \begin{columns}[t]
    
%%%%%%%%%%%%%%%%%%%%%%%%%%%%%%%%%%%%%%%%%%%      
    
      \begin{column}{.32\linewidth}
      \centering
        \begin{block}{Enigma}
          The Enigma is an electromechanical device that was the main force behind the impressive interconnectivity available to the Axis powers throughout the Second World War.
        \end{block}
        
         \begin{block}{The Antique Enigma}
          \includegraphics[width=.5\columnwidth]{historicEnigma.png} 
        \end{block}
        
         \begin{block}{History} \small{
          The Enigma was the brainchild of  Arthur Scherbius at the end of the First World War; at which point, the first uses were purely commercial. In February of 1918 Mr Scherbius along with Mr Ritter applied for a patent for a machine that used rotors for the purposes of encryption. This was the first implementation of the Enigma. The Enigma has many different versions, this includes the version that was used by the German Navy also known as the M3. The M3 incorporates three of the patented rotors that are the hallmark of the machine. Aesthetically, the machine was very similar to a large typewriter. The M3 has a keyboard that was used as an input for the message to be encrypted, and a light-board to show the encrypted letter. The settings used by the network of machines was sent to all operators. They would then set up the machine by selecting the three rotors in the order in which they would be placed into the machine. Operators would also reprogram the plug-board and the reflector. Once the machine was set up it was possible to encrypt and decrypt message that were sent using the settings provided.}
        \end{block}
        
        \begin{block}{Technical Diagram of the Enigma}
          \includegraphics[width=\columnwidth]{enigma.png} 
        \end{block}
        
         \begin{block}{Technical}
          \begin{itemize} \small{
          \item The M3 is made up of a plug-board, three rotors and a reflector.
          \item The reprogrammable plug-board is a simple mapping between two alphabets.
          \item The rotors are a stepping map between two alphabets that feed into the following rotor.
          \item The reflector is very similar to the plug-board and feeds its output back into the third rotor for the return journey.
          \item The whole system comes together in the form of a letter being input into the plug-board, which will then map from the input alphabet to the one specified by the settings of the plug-board, then these will be input into the first of the three rotors, this will perform another mapping, output into the second rotor, another mapping, input into the third rotor which would then output to the reflector. The reflector is another static mapping that will not change until the settings used for each day are changed, unlike the rotors which change mapping with each letter of the message that is input. Once the reflector has mapped its input to the output, it is then it put back into the third rotor but is mapped in the opposite direction. This is then repeated in the second and first rotors, in that order. Finally the letter is returned through the plug-board, then output to the light-board.}
          \end{itemize}
        \end{block}

      \end{column}

%%%%%%%%%%%%%%%%%%%%%%%%%%%%%%%%%%%%%%%%%%%      

      \begin{column}{.32\linewidth}
      \centering
       \begin{block}{BOMBE}
          The BOMBE was an electromechanical device used by the British with help from the Polish to break the Enigma codes used during the Second World War.
        \end{block}
        
        \begin{block}{A historic photograph of the BOMBE}
          \includegraphics[width=\columnwidth]{historicBOMBE.png} 
        \end{block}
        
        \begin{block}{History} \small{
		First conceptualised by Polish cryptographer Marian Rejewski as part of the Biuro Szyfr\'ow or Cipher Bureau in 1938. This was known as the bomba kryptologiczna or cryptologic bomb. This was the first stage of the breaking of the Enigma. In 1939 a revised and much improved version was proposed by Alan Turing, with another breakthrough being made by Gordon Welchman in 1940. The mechanical engineering and construction work was done by Harold Keen of the British Tabulating Machine Company throughout the life of the BOMBE and its developments.}
        \end{block}
        
         \begin{block}{The cabling of the BOMBE}
          \includegraphics[width=\columnwidth]{bombeCable.png} 
        \end{block}
        
        \begin{block}{Technical}
          \begin{itemize}\small{
          \item The BOMBE was made up of three banks of twelve columns of three rotor representations.
          \item The BOMBE was was made to be a representation of the core of the Enigma, meaning the three rotors and reflector. Each bank would be wired to test a specific setting and would stop if the setting was tested to be correct. This test was done by a component known as a diagonal-board, developed by Gordon Welchman.
          \item The rotor that represented rotor one of the Enigma rotated at a speed of 120rpm, and each subsequent rotor at 1/26th the speed of its predecessor.
          \item Once a "Stop" had been reached, meaning that the settings had been tested and were a possible match to the settings used by the German forces that day. Following from this a frequency analysis would be performed on the results to verify that these were in fact the correct settings.}
          \end{itemize}
        \end{block}

      \end{column}
      
 %%%%%%%%%%%%%%%%%%%%%%%%%%%%%%%%%%%%%%%%%%%      
      
      \begin{column}{.32\linewidth}
        \begin{block}{Parallel Techniques}
          Throughout history, multitasking has been a dream by mankind worldwide. It is called parallel programming in computers and has been possible commercially since the turn of the millennia.
        \end{block}
        
        \begin{block}{History}
          The idea of Multiple Instructions Multiple Data, or truly independent parallel programming was discussed in 1958 by S. Gill and by IBM researchers John Cocke and Daniel Slotnick in the same year. The first processor that was capable of parallel programming was introduced in 1962 by the Burroughs Corporation in the form of the D825. Other Single Instruction Multiple Dataset processors similar to the D825 can also be traced from around this time. The first true commercial processors were not available until nearer to the turn of the century with AMD and Intel releasing their first multicore processors to the public around the same time.
        \end{block}
        
        \begin{block}{SIMD and MIMD}
          \includegraphics[width=\columnwidth]{parallel.png} 
        \end{block}
        
        \begin{block}{Technical}
          \begin{itemize}
          \item The idea of the MIMD system is performing different instructions on different sets of data concurrently.
          \item The idea of the SIMD system is performing the same instructions on multiple different pieces of data concurrently, also known as vector processing.
          \item The MIMD architecture is based around the concept of separate cores, each of which have separate memory but also have shared memory to which most of the data is stored. These have a set of rules that dictate how this memory is accessed and written to between the cores.
          \item The SIMD architecture uses vectorised loads - the idea that more than one data item can have an operation applied to it at once.
          \end{itemize}
        \end{block}
        
	\begin{block}{Problem Decomposition}
          \includegraphics[width=\columnwidth]{problemDecomp.png} 
        \end{block}

      \end{column}      
      
%%%%%%%%%%%%%%%%%%%%%%%%%%%%%%%%%%%%%%%%%%%  
      
    \end{columns}

  \vfill
    \begin{block}{\centering Results}
		The time taken by the historic version of the BOMBE to decrypt a 20 letter ciphertext was roughly 14400 seconds; approximately 4 hours. The time taken for a brute force version of the BOMBE implemented on a modern processor takes \num{6.12e47} seconds or in the range of \num{6e40} years to perform the same task. On average, the recreation version of the BOMBE takes 48.22 seconds and the improved parallel version took only 17.62 seconds to break the Enigma code.
    \end{block}
    \vfill

  \end{frame}
\end{document}


